% !TEX root=paper.tex
\section{Introduction}\label{introduction}

Multi-layered convolutional models (CNNs) have recently demonstrated impressive levels of performance on visual detection tasks, by applying deep CNNs trained on object classification data to multiple candidate image windows in a scene \cite{Sermanet-ICLR-2014,Girshick-CVPR-2014}.
While convolutional models are efficient to implement due to the availability of high-performance primitives, the amount of work to perform complete inference at all candidate region windows is still considerable.

The standard method for quickly applying a CNN to a large number of candidate regions is to restrict the regions to lie in a regular scale-space pyramid.
With this restriction, the CNN can efficiently compute features for all regions using convolution.
This approach was used in classical CNN-based detection systems as well as recent approaches such as OverFeat \cite{Sermanet-ICLR-2014}.
The recently proposed Regions with CNN features (R-CNN), in contrast, allows for more general region proposals guided by low-level segmentation cues (\cite{Uijlings-IJCV-2013}), but repeats the computation of convolutional features across numerous overlapping windows without sharing computation.

In this paper, we investigate a novel approach for speeding CNN-based detection methods and propose a general technique for accelerating CNNs applied to class imbalanced data (such as in object detection datasets).
We bring the classic idea of the cascade to CNNs by inserting a \emph{reject} option between CNN layers.
When the CNN processes batches of images, which is standard for many applications, the reject layers allows the CNN to ``thin'' the batch as it progresses through the network, thus saving processing time.
