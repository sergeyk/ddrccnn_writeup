Recently, an object detection architecture with two key points has demonstrated an undeniable increase in performance on the most challenging detection datasets.
The first point is doing bottom-up Region of Interest (ROI) proposals instead of sliding-window evaluation.
The second point is using multi-layer Convolutional Neural Nets (CNNs) pre-trained on large-scale classification data to featurize canonically resized ROIs.
Although state of the art in recognition performance, the method is computationally costly, requiring CNN processing of upwards of two thousand regions.
Even with efficient GPU implementation, the method takes \textasciitilde{}10 s per image after the computation of region proposals.% (another \textasciitilde{}2 s).

We propose and evaluate several novel speed-ups for the Region-CNN (R-CNN):
(1) An initial region filtering step using a ``fast feature,'' with dynamic updates.
(2) Pyramid R-CNN ``fast feature'': the whole image is processed with a CNN up to the highest
non-fully-connected layer, from which ROIs are featurized by cropping and resizing.
(3) Cascaded-CNN: each ROI passing the initial filter is further processed by a CNN that is augmented with a \emph{reject} option after each layer.

We achieve a \textasciitilde{}8x speed-up while losing no more than 10\% of the peak R-CNN detection performance -- or a \textasciitilde{}16x speed-up while losing no more than 20\%.
