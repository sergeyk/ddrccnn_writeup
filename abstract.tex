%!TEX root=paper.tex
\begin{abstract}
{\bf TD: this para reads awkward and a bit arrogant to me; can we replace with one crisp and strong sentance?  I also hate two paragraph abstracts.}
Recently, an object detection architecture with two key points has demonstrated an undeniable increase in performance on the most challenging detection datasets.
The first point is doing bottom-up Region of Interest (ROI) proposals instead of sliding-window evaluation.
The second point is using multi-layer Convolutional Neural Nets (CNNs) pre-trained on large-scale classification data to featurize canonically resized ROIs.
Although state of the art in recognition performance, the method is computationally costly, requiring CNN processing of upwards of two thousand regions.
Even with efficient GPU implementation, the method takes \textasciitilde 10 s per image after the computation of region proposals.% (another \textasciitilde 2 s).

We propose and evaluate several novel speed-ups for the Region-CNN (R-CNN):
(1) Initial or ongoing region filtering using one of three ``fast features'': (a) region overlap statistics; (b) CNN pixel gradient statistics; and (c) the \emph{Pyramid R-CNN}, which processes the whole image up to the highest non-fully-connected layer before cropping, resizing, and classifying ROIs.
(2) \emph{Cascaded CNN}: each ROI passing the initial filter is further processed by a CNN that is augmented with a \emph{reject} option after each layer, allowing much faster but accurate classification.

We achieve a \textasciitilde 8x speed-up while losing no more than 10\% of the top R-CNN detection performance -- or a \textasciitilde 16x speed-up while losing no more than 20\%.
\end{abstract}
