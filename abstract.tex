Recently, an object detection architecture based on two key decisions
has demonstrated an undeniable increase in performance on both PASCAL
VOC and ILSVRC detection datasets. The first is doing bottom-up Region
of Interest (ROI) proposals instead of sliding-window evaluation. The
second is using multi-layer Convolutional Neural Nets (CNNs) pre-trained
on ILSVRC classification data to featurize canonically resized ROIs.
Although by far the best in terms of recognition performance, the method
is computationally costly, requiring processing of several hundred of
thousand regions with the CNN. Even with efficient GPU implementation,
the method takes \textasciitilde{}10 s per image after the computation
of region proposals (another \textasciitilde{}2 s).

We propose three novel speed-ups for the task. (1) Dense evaluation: the
whole image is processed with a CNN up to the highest
non-fully-connected layer, and ROIs are featurized by cropping and
resizing that highest layer, for use as an initial filter. (2) Cascaded
CNN: each ROI passing the initial filter is further processed by a CNN
that is augmented with a \emph{reject} option after each layer. (3) The
selection of ROIs for further processing is done dynamically, taking
into account the evaluation results on the ROIs selected so far.

Combined, these speed-ups allow us to match original R-CNN performance
in 20\% of the time.
