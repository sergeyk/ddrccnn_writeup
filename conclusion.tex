%!TEX root=paper.tex
\section{Conclusion}\label{sec:conclusion}

We have presented approaches to speed up region-based convolutional
object detectors. Our approach is based on the reprioritization of
candidate regions so as to maximize mAP within a given time budget.
We consider techniques based on both re-ranking and
rejection-filtering based on lightweight features of the observed
image region.  We re-order regions to process the most promising ones
first using one of three ``fast features'' based on region overlap
statistics, CNN pixel gradient statistics, and a convnet feature
pyramid.  While our work is generally applicable to any windowed
detection, we specifically develop a novel cascaded approach for
speeding up convolutional neural networks and evaluate it on the
recently introduced R-CNN object detection system, which has excellent
detection accuracy but is realtively slow. We show how to effecively
amortize the time R-CNN spends classifying regions of interest by
sharing computation between overlapping regions, and pruning regions
unlikely to be valuable.  Our approach introduces a \emph{reject}
option between layers in the CNN, allowing the network to terminate
computation early, as in a traditional cascade.  We reported
significant speed-ups while losing a modest amount of detection
performance. Our scheme allows R-CNN to be used in time-critical
applications.

