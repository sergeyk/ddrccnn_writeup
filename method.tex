%!TEX root=paper.tex
\section{Method}\label{sec:method}

We take the Region-CNN (R-CNN) \cite{Girshick-CVPR-2014} method as our point of departure.
As summarized in \autoref{fig:rcnn}, R-CNN starts with external region-of-interest proposals (ROIs).
ROIs are taken in batches: transformed to canonical size, and classified with a CNN, obtaining multi-class scores for each region.
After all batches have been scored, ROIs are post-processed with non-maximal suppression and other bounding box refinement to obtain the final detections.

We start with the same ROI proposals, but first score all proposals with a quick-to-compute feature.
We then select a batch of highly-scoring ROIs and classify them with the CNN -- which can additionally be sped up with cascaded structure.
After the batch is scored, we optionally re-score the remaining ROIs, and repeat the process until we're either out of time, or out of regions.
\autoref{fig:combined} summarizes the process.

For the quick-to-compute feature, we consider statistics about ROI location and overlap with other regions, as described in \autoref{sec:region}; whole-image CNN pixel gradient in \autoref{sec:gradient}, and an amortized R-CNN variant that operates on a feature pyramid in \autoref{sec:dense}.
The Cascaded CNN is described in \autoref{sec:ccnn}.
The region selection is explained in \autoref{sec:dynamic}.

\begin{figure}[h!]
\begin{center}
\includegraphics[width=0.98\columnwidth]{figures/rcnn.pdf}
\caption{
R-CNN architecture: image regions are cropped, resized, and each one fed through a CNN with classification layers.
The classifier outputs are post-processed to give the final detections.
}\label{fig:rcnn}
\end{center}
\end{figure}

\begin{figure}[h!]
\begin{center}
\includegraphics[width=0.98\columnwidth]{figures/combined.pdf}
\caption{
Our method scores each region of interest with a fast feature (we evaluate several), allowing us to pick promising regions first.
The regions are classified with the original CNN, or a sped-up Cascade-CNN.
The properties of the regions can play a role in selecting the next batch of regions.
}\label{fig:combined}
\end{center}
\end{figure}

%%%%%%%%%%%%%%%%%%%%%%%%%%%%%%%%%%%%%%%%%%%%%%%%%%%%%%%%%%%%%%%%%%%%%%%%%%%%%%%
\subsection{Region statistics}\label{sec:region}

We compute a very simple feature for each ROI: its normalized location,
its scale($\sqrt{\text{width} \times \text{height}}$),
aspect ratio ($\log \sqrt{\text{width} / \text{height}}$),
and normalized counts of overlapping regions, at different PASCAL overlap \todo{definition} thresholds (0, 0.2, 0.6).

This simple feature works suprisingly well for filtering regions to process first, and is always included in concatenation with the more complex features that are described next.

%%%%%%%%%%%%%%%%%%%%%%%%%%%%%%%%%%%%%%%%%%%%%%%%%%%%%%%%%%%%%%%%%%%%%%%%%%%%%%%
\subsection{CNN pixel gradient computation}\label{sec:gradient}

One fast-to-compute feature we consider for the preliminary featurization of ROIs is the pixel gradient, back-propagated throgh the classifier CNN applied to the whole image.
This gradient corresponds to a kind of saliency map onto the image \cite{Simonyan-ICLR-2014}, and can therefore be used to evaluate ROIs.

We have fine-tuned an ``AlexNet'' \cite{Krizhevsky-NIPS-2012} CNN using the PASCAL VOC 2007 classification challenge.
Unlike the ILSVRC, the PASCAL VOC is a multi-label prediction task, with at times multiple correct labels for an image.
Accordingly, we implement a simple multi-label loss layer for this fine-tuning.
\todo{@sergio: details needed?}

This pixel-wise gradient can be seen as a first-order approximation to the weight of a pixel in the classification output.
To compute it, we first run the CNN forward from image input to the prediction layer.
Back-propagating through the model and a last backward pass to the input gives the pixel-wise gradient.
A couple of example images are shown in \autoref{fig:gradient}.

We compute an integral image on this pixel gradient map, allowing near-instant computation of sums in arbitrary regions.
For each region to be evaluated, we compute the total gradient sum, sums for each of the corners, and proportions of in-region vs. out-of-region sums.


%%%%%%%%%%%%%%%%%%%%%%%%%%%%%%%%%%%%%%%%%%%%%%%%%%%%%%%%%%%%%%%%%%%%%%%%%%%%%%%
\subsection{Pyramid amortized R-CNN}\label{sec:dense}

Another fast ROI featurization we consider is a variant of R-CNN that processes the whole image, at multiple scales, with the CNN up to the topmost non-fully-connected layer.
We call this multiscale CNN output a ``feature pyramid''.
As \autoref{fig:dense_rcnn} shows, ROIs are then cropped from the feature pyramid at the best matching scale, warped to a canonical size, and classified.
ROIs are highly overlapping, but their featurization is shared through the pyramid, amortizing its construction cost.
This doesn't work as well as the original R-CNN system, but is much faster, and can therefore be used as the fast feature for the dynamic region selection.

% \subsubsection{Design choices}

We explored a variety of choices while designing Pyramid R-CNN.
We tried using a single scale or a pyramid with 7 levels each separated by a scale factor of $2^{-1/2}$.
For warping, we experimented with canonical sizes of $s \times s$, for $s \in \{5,6,7\}$ and resampling with nearest neighbor or bilinear interpolation.
We also tried two variants of the feature pyramid: the raw feature pyramid or one where each level is max pooled with a $3 \times 3$ pooling window run at a stride of $1$ (to avoid subsampling).

\autoref{tab:prcnn} shows mAP performance on PASCAL VOC 2007 test for these various choices.
The best configuration, in terms of mAP, uses a 7 level pyramid, a warp size of $7 \times 7$ with bilinear interpolation, and max pooling.
The first level of our pyramid is computed from a $1713 \times 1713$ pixel image, which yields a $108 \times 108$ cell feature map.
The gold standard (non-pyramid) R-CNN performance using the same non-fine-tuned CNN is 44.2\% mAP.
Our best result with Pyramid R-CNN is slightly worse, at 41.9\%.

To map a ROI $R$ into the pyramid, we start by computing an ``optimal'' scale defined by $\alpha^* = 227/\min(h,w)$, where $h$ and $w$ are image height and width of $R$.
We then find the nearest level in the pyramid: $l^* = \textrm{argmin}_l |\log(\alpha_l) - \log(\alpha^*)|$, where $\alpha_l$ is the scale factor for pyramid level $l$.
This procedure approximates the scale that R-CNN would use, since it operates on $227$ pixel inputs.

\begin{table}[ht]
\centering
\caption{
    Pyramid R-CNN design choices, vs mAP performance.
    We additionally show two best and two worst performing classes.
}\label{tab:prcnn}
\small{
\begin{tabular}{@{}rccccl@{}}
\toprule
settings      & \mcell{warp 7x7\\nearest}  & \mcell{warp 7x7\\nearest}  & \mcell{warp 7x7\\bilinear}  & \mcell{max pooled 3x3\\warp 7x7\\bilinear} \\
scales        & 1                          & 7                           & 7                          & 7 \\
\midrule
% aeroplane     & 31.7                       & 46.8                        & 44.5                       & 47.0 \\
bicycle       & 36.1                       & 50.1                        & 52.6                       & 57.9 \\
% bird          & 12.3                       & 25.4                        & 26.7                       & 31.1 \\
% boat          & 12.5                       & 20.4                        & 25.7                       & 27.5 \\
% bottle        & 11.4                       & 12.9                        & 12.6                       & 19.8 \\
% bus           & 17.1                       & 43.6                        & 45.4                       & 50.7 \\
car           & 39.8                       & 54.3                        & 56.7                       & 60.0 \\
% cat           & 7.0                        & 39.0                        & 42.4                       & 48.2 \\
chair         & 6.7                        & 11.7                        & 11.8                       & 16.0 \\
% cow           & 25.8                       & 41.9                        & 44.1                       & 50.1 \\
% diningtable   & 10.2                       & 29.2                        & 33.0                       & 38.6 \\
% dog           & 7.2                        & 35.3                        & 36.0                       & 41.6 \\
% horse         & 17.3                       & 46.3                        & 51.2                       & 55.8 \\
% motorbike     & 27.8                       & 49.4                        & 53.3                       & 56.6 \\
% person        & 16.6                       & 28.8                        & 31.3                       & 36.4 \\
pottedplant   & 11.4                       & 17.1                        & 18.6                       & 20.9 \\
% sheep         & 19.3                       & 34.4                        & 35.8                       & 40.2 \\
% sofa          & 10.3                       & 22.8                        & 29.3                       & 35.8 \\
% train         & 17.9                       & 38.1                        & 43.0                       & 44.9 \\
% tvmonitor     & 36.3                       & 50.0                        & 54.0                       & 59.7 \\
\midrule
mAP           & 18.7                       & 34.9                        & 37.4                       & 41.9 \\
\bottomrule
\end{tabular}

}
\end{table}

\begin{figure}[h!]
\begin{center}
\includegraphics[width=0.98\columnwidth]{figures/dense_rcnn.pdf}
\caption{
Post R-CNN architecture: the whole image is fed through a CNN up to the highest pooling layer.
Regions are cropped from that layer at the best matching scale, resized, and classified.
}\label{fig:dense_rcnn}
\end{center}
\end{figure}

%%%%%%%%%%%%%%%%%%%%%%%%%%%%%%%%%%%%%%%%%%%%%%%%%%%%%%%%%%%%%%%%%%%%%%%%%%%%%%%
\subsection{Cascaded CNN}\label{sec:ccnn}

Most ROIs that go through the CNN in R-CNN do not contain objects of interest, and are simply background.
It would be useful to quickly reject these regions, without expending the full amount of computation on them. Since most of the time is spend in the convolution layers, we introduce a reject layer after pool1, pool2, conv3 and conv4, this allows skipping the computation of the following convolutions. These reject layers were trained using the corresponding fine-tuned CNN.

The Cascaded CNN, shown in \autoref{fig:ccnn}, augments the CNN network with an ``Early Reject'' option: after some layers, the network decides whether to keep computing the input with the next layer, or to reject the region since it is background.
For our purpose, the reject layers are trained to separate Background/Foreground, terminating if they are confident that the input is Background. The last classification layer still outputs the full multi-class scores for the surviving regions.

\begin{itemize}
\itemsep1pt\parskip0pt\parsep0pt
\item
  \todo{Explain training the thresholds.}
\end{itemize}

\lorem{Lorem ipsum dolor sit amet, consectetur adipisicing elit, sed do eiusmod tempor incididunt ut labore et dolore magna aliqua. Ut enim ad minim veniam, quis nostrud exercitation ullamco laboris nisi ut aliquip ex ea commodo consequat. Duis aute irure dolor in reprehenderit in voluptate velit esse cillum dolore eu fugiat nulla pariatur. Excepteur sint occaecat cupidatat non proident, sunt in culpa qui officia deserunt mollit anim id est laborum.}

\begin{figure}[h!]
\begin{center}
\includegraphics[width=0.98\columnwidth]{figures/ccnn.pdf}
\caption{
The Cascaded CNN has a reject option after pool1, pool2, conv3 and conv4 layers.
\todo{Redo figure to introduce the Reject layers and color their arrows red to make it clear that a choice is being made.}
}\label{fig:ccnn}
\end{center}
\end{figure}

%%%%%%%%%%%%%%%%%%%%%%%%%%%%%%%%%%%%%%%%%%%%%%%%%%%%%%%%%%%%%%%%%%%%%%%%%%%%%%%
\subsection{Region selection}\label{sec:dynamic}

As shown in \autoref{fig:combined}, the quick-to-compute features are used to select batches of regions in a way that orders the regions most likely to contain ground truth objects earlier than other regions.
Unlike the Cascade-CNN, this process of region selection does not reject regions flat out, but merely reorders them to be processed later.

We consider two ways of selecting regions by their score.
The first is to simply sort the regions by score, and to take batches in that sorted order
We quickly discovered that this naive method does not result in good performance, and in fact can be worse than taking regions in an entirely random order.
We hypothesized that this is because the highest scoring regions may all


\subsubsection{Dynamic feature update}
