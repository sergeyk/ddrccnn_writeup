%!TEX root=paper.tex
\section{Related Work}\label{sec:related}

\paragraph{Object recognition with CNN}\label{object-recognition-with-cnn}
The recent success of deep convolutional neural networks (CNN) Alexnet \cite{Krizhevsky-NIPS-2012} on image classification tasks such as ImageNet \cite{deng2009cvpr} has spurred various investigations \cite{Girshick-CVPR-2014,Zou-CVPR-2014,Simonyan-ICLR-2014,Sermanet-ICLR-2014} how this success can be carried over to an detection task despite the prohibitive cost of evaluating such models densely in a naive manner. These previous attempts aim at reducing computational cost by reducing the search space of sliding window search by a bottom up window proposal scheme \cite{Girshick-CVPR-2014}, reducing computational overhead by sharing computation between windows to be evaluated \cite{Zou-CVPR-2014,Sermanet-ICLR-2014} as well as deriving a window sampling scheme that is based on the gradient of the network \cite{Simonyan-ICLR-2014}. All these method have in common that they rely on a fixed strategy and cannot adapt the computational scheme based on intermediate results, while we propose a dynamic scheme that steers computation based on intermediate outcomes in order to spend computation time most effectively. In contrast to previous work on shared dense computation layers in CNN classifiers for detection \cite{Zou-CVPR-2014,Sermanet-ICLR-2014}, we maintain effective canonicalization of region via warping and show its importance for achieving high performance.

\paragraph{Cascaded detection}\label{cascaded-detection}
Detection cascade \cite{Viola2004,Felzenszwalb-CVPR-2010} is a well established techniques to condition the computational scheme on intermediate outcomes based on a fixed evaluation order to features/classifiers. This idea has seen several refinements including sharing computation across cascades \cite{Dollar-ECCV-2012} and the possibility to skip evaluations \cite{benbouzid12icml}. Recently, a CNN with a cascaded structure for coarse to fine inference of facial feature points has been proposed \cite{cnn_cascade}. While previous cascade schemes really on a fixed ordering of evaluations, our scheme learns to traverse the available features/classifiers in arbitrary orders.

\paragraph{Dynamic selection}\label{dynamic-selection}
Most recently
  Timely Object Recognition \cite{Karayev-NIPS-2012}
  Cross-talk cascades \cite{Dollar-ECCV-2012}

