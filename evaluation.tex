%!TEX root=paper.tex
\section{Evaluation}\label{sec:evaluation}

We evaluate on the standard object detection benchmark: the PASCAL VOC \todo{cite?}.
In all cases, the CNN region classifiers are trained on the PASCAL VOC 2007 trainval set.
The parameters of our methods are set by training or cross-validation on the VOC 2007 val set.
We evaluate on the VOC 2007 test set.

The R-CNN software was used as available in June 2014\footnote{\url{https://github.com/rbgirshick/rcnn}}.
That software relies on Selective Search \cite{Uijlings-IJCV-2013} region proposals.
Different images are proposed different numbers of regions.
\autoref{fig:roi_hist} shows the distribution of number of regions on the validation set, with the parameters of the R-CNN.

An additional parameter is the size of each batch of regions that goes through the CNN.
We set batch size to 100 regions, and observe that it takes on average 500 ms to process them with the CNN.
In all experiments, we use Ubuntu 12.04, Intel i5 3.2GHz CPU, and NVIDIA Tesla K40 GPU.

\autoref{fig:voc2007_results} and \autoref{tab:results} show the results.

\begin{figure}[ht]
\centering
\begin{subfigure}[b]{0.52\linewidth}
    \includegraphics[width=\linewidth]{figures/_apvst_final.pdf}
    \caption{mAP vs. time allotted for detection}\label{fig:apvst}
\end{subfigure}\hfill
\begin{subfigure}[b]{0.45\linewidth}
    \includegraphics[width=\linewidth]{figures/_speedup_final_abs.pdf}
    \caption{mAP vs. speed-up factor}\label{fig:speedup}
\end{subfigure}
\caption{
Results on the PASCAL VOC 2007 dataset.
(a) On the left-hand mAP vs. Time plot, we can compare APs at a given allotted time point.
For example, at 1300 ms, random region selection gets about 0.42 mAP, while our best method (C-CNN with gradient-based region selection) obtains 0.50 mAP.
(b) On the right-hand speed-up plot, we can compare speed-ups at a given mAP point.
For example, we can see that we should obtain mAP of 0.40 at around 20x speedup with our method.
}\label{fig:voc2007_results}
\end{figure}

\begin{table}[ht]
\centering
\caption{
Full table of AP vs. Time results on PASCAL VOC 2007.
}\label{tab:results}
\small{
\begin{tabular}{lrrrrrrrr}
\toprule
Time allotted (ms) &  0     &  300   &  600   &  1300  &  1800  &  3600  &  7200  &  10000 \\
\midrule
Original                            &  0.000 &  0.000 &  0.176 &  0.211 &  0.244 &  0.368 &  0.496 &  0.544 \\
Random                              &  0.000 &  0.000 &  0.295 &  0.381 &  0.426 &  0.504 &  0.536 &  0.544 \\
Region Selection                    &  0.000 &  0.000 &  0.412 &  0.463 &  0.485 &  0.526 &  0.540 &  0.544 \\
Region Selection w/ Gradient        &  0.000 &  0.000 &  0.424 &  0.469 &  0.490 &  0.526 &  0.542 &  0.544 \\
Region Selection w/ P-RCNN          &  0.000 &  0.000 &  0.000 &  0.457 &  0.498 &  0.537 &  0.544 &  0.544 \\
C-CNN                               &  0.000 &  0.327 &  0.430 &  0.493 &  0.510 &  0.530 &  0.528 &  0.544 \\
C-CNN, Region Selection w/ Gradient &  0.000 &  0.198 &  0.442 &  0.502 &  0.517 &  0.528 &  0.528 &  0.544 \\
C-CNN, Region Selection w/ P-RCNN   &  0.000 &  0.000 &  0.000 &  0.451 &  0.503 &  0.527 &  0.528 &  0.544 \\
\bottomrule
\end{tabular}

}
\end{table}

\begin{description}
  \item[Original] \hfill \\
  The original order of the Selective Search regions of interest.
  This order is influenced by the hierarchical segmentation of their method, and so has sequences of highly overlapping regions.
  \item[Random] \hfill \\
  A completely blind permutation of the original order.
  \item[Region Selection] \hfill \\
  The order is determined as described in \autoref{sec:dynamic}.
  The region statistics feature from \autoref{sec:region} is always used, allowing learning an effective ordering.
  Additionally, we consider the Pixel Gradient and the Pyramid-R-CNN features.
\end{description}

\begin{figure}
\centering
\begin{subfigure}[b]{0.52\linewidth}
    \includegraphics[width=\linewidth]{figures/roi_hist.pdf}
    \caption{Distribution of number of regions per image.}\label{fig:roi_hist}
\end{subfigure}\hfill
\begin{subfigure}[b]{0.46\linewidth}
    \includegraphics[width=\linewidth]{figures/gradient.pdf}
    \caption{Example of the CNN gradient.}\label{fig:gradient}
\end{subfigure}
\caption{}
\end{figure}
